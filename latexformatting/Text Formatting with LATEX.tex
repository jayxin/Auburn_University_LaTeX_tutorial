\documentclass{beamer}

%\usepackage{beamerthemeAUTheme}
%\usepackage{beamerthemesplit}
\usetheme[hideothersubsections]{AUTheme}
%\useoutertheme{Boadilla}
%\usetheme{Madrid}
\usefonttheme[onlymath]{serif}

\usepackage{amsmath, latexsym, color, graphicx, amssymb, here}
\usepackage{epsf, epsfig, pifont,tikz,subfigure}
\usepackage{graphics, calrsfs}
\usepackage{times}
\usepackage{fancybox,calc}

\newcommand{\parD}[2]{\frac{\partial #1}{\partial #2}}
\newcommand{\parDD}[2]{\frac{\partial^2 #1}{\partial #2 ^2}}
\newcommand{\laplacian}{\Delta}
\renewcommand{\div}{\nabla\cdot}
\newcommand{\grad}{\nabla}
\newcommand{\divp}{\nabla^\prime\cdot}
\newcommand{\gradp}{\nabla^\prime}
\newcommand{\curl}{\nabla\times}
\newcommand{\cross}{\times}
\renewcommand{\dot}{\cdot}
% define some colors
\definecolor{cBlue}{rgb}{.255,.41,.884} % RoyalBlue of svgnames
\definecolor{cRed}{rgb}{1, 0, 0} % Red of svgnames



%\usecolortheme[named=blue]{structure}

\title{Basic Text Formatting with \LaTeX}
\author{Wenting Deng}
\institute{Department of Electrical Engineering}
\date{June 30th 2010}

\begin{document}

%\institution{Department of Electrical and Computer
%Engineering \\  Auburn Chapter of SIAM  \\
%Auburn University}


\frame{\titlepage}
%\slideCaption{\LaTeX}

%------------------------------------------------------------Slide 1

\frame {\frametitle{Special Characters}

\begin{itemize}

\item Single quotation marks: left `\ and right\ ' \\
\item Double quotation marks: two single left quotes `\ ` \ and two single right quotes \ '\ '\ or the double quote key (")\\
\item A double quote followed by a single quote, or vice-versa: command $\backslash$,\ between two quotation marks\\[10pt]
{\small For example}:{\scriptsize \textit { ``\,`Fi' or `fum'\,''he asked }\\}
\end{itemize}
}

%------------------------------------------------------------Slide 2
\frame {\frametitle{Special Characters}
There are three kinds of dashes in typeset documents by typing one, two, or three ``-'' characters:\\[12pt]
{\scriptsize \texttt {An intra-word dash, as in X-ray}\\}
An intra-word dash, as in X-ray\\[7pt]
{\scriptsize \texttt {A medium dash for number range, like 1-\ -2}\\}
A medium dash for number range, like 1--2\\[7pt]
{\scriptsize \texttt {A punctuation dash-\ -\ -like this}\\}
A punctuation dash---like this\\
}

%------------------------------------------------------------Slide 3

\frame {\frametitle{Special Characters}


\begin{figure}[h!]
\centering
\includegraphics[width=5cm]{special.jpg}
\end{figure}
}
%%--------------------------------------------------------------------------Slide 4
\frame {\frametitle{Special Characters}


\begin{figure}[h!]
\centering
\includegraphics[width=7cm]{special1.jpg}
\end{figure}
}
%%--------------------------------------------------------------------------Slide 5



\frame {\frametitle{Type Style}


\begin{figure}[h!]
\centering
\includegraphics[width=7cm]{types.jpg}
\end{figure}

These commands can be combined, provided the font thus requested actually exists. \\[10pt]
{\small For example}: \texttt{\scriptsize $\backslash$textbf\{$\backslash$textit\{This is bold italic\}\}} \\
produces: \textbf{\textit{This is bold italic}}.

~\\
}

%%--------------------------------------------------------------------------Slide 6
\frame {\frametitle{Type Size}


\begin{figure}[h!]
\centering
\includegraphics[width=7cm]{size.jpg}
\end{figure}
\begin{itemize}
\item The actual size produced by each command depends on the initial point size selected for the document\\
\item The size-changing commands are usually used within a group (i.e., braces) to delimit the range of their action\\
\item To change both type size and style at the same time, commands can be used together\\
\end{itemize}
}
%%--------------------------------------------------------------------------Slide 7



\frame {\frametitle{Sentences and Paragraphs}


\setbeamercolor{uppercol}{fg=white,bg=ta3orange}%
\setbeamercolor{lowercol}{fg=black,bg=taorange}%
\begin{itemize}
\item \TeX\ ignores the way the input is formatted, but pay attention only to the logical concepts end of sentence and end of paragraphy.\\
\item The command ``$\backslash$ $\backslash$'' will force a new line. \\
\item A value inside square brackets following ``$\backslash$ $\backslash$'' will specify the amount of blank space between lines.\\
\item The command to force a new page is \texttt{$\backslash$newpage}.\\
\end{itemize}

~\\
}
%------------------------------------------------------------Slide 8

\frame {\frametitle{Sentences and Paragraphs}

\textit{\Large Example}:

\texttt{\scriptsize The ends of \ \ \ \ \ words and sentences are marked by spaces you type. \ \ \ It doesn't matter how many spaces you type; one is as good as 100.}\\
The ends of words and sentences are marked by spaces you type. It doesn't matter how many spaces you type; one is as good as 100.\\[20pt]

This Line will be 20pt away from previous sentence.

}

%------------------------------------------------------------Slide 9

\frame {\frametitle{Preventing Line Breaks}
\begin{itemize} 
\item Line breaking should be prevented at certain interword spaces.\\[7pt]
{\scriptsize \textit {For example, the expression ``Chapter 3'' looks strange if the ``Chapter'' ends one line and the``3'' begins the next.}}\\
\item A tilde character \~{} produces an ordinary interword space at which \TeX\ will never break a line.
\end{itemize}

\texttt{\scriptsize Mr.\~{}Jones }\ \ \ Mr.~Jones\\
\texttt{\scriptsize U.\~{}S.\~{}Grant }\ \ \ U.~S.~Grant\\


}
%%--------------------------------------------------------------------------Slide 10

\frame {\frametitle{Horizontal and Vertical Space} 
\begin{itemize}
\item If you want to leave horizontal and vertical space in your text, use commands \texttt{$\backslash$hspace*} and \texttt{$\backslash$vspace*}.\\
\item If you want to center one line, use the command \{\texttt{$\backslash$centerline}...\}\\
\end{itemize}


\texttt{\scriptsize \{$\backslash$centerline \ This line will be centered.\}}\\[5pt]
\centerline{This line will be centered.}

\texttt{\scriptsize $\backslash$hspace*\{1in\} This text starts from a one-inch space.}\\
\hspace*{1in}This text starts from a one-inch space.


}

%------------------------------------------------------------Slide 11

\frame { \frametitle{Lists}
\begin{figure}[h!]
\centering
\includegraphics[width=10cm]{list1.jpg}
\end{figure}

}
%------------------------------------------------------------Slide 12
\frame { \frametitle{Lists}
\begin{figure}[h!]
\centering
\includegraphics[width=10cm]{list2.jpg}
\end{figure}

}
%------------------------------------------------------------Slide 13
\frame { \frametitle{Lists}
Example of nesting list environments
\begin{figure}[h!]
\centering
\includegraphics[width=10cm]{list3.jpg}
\end{figure}

}

%%-----------------------------------------------------------------------------Slide 14
\frame { \frametitle{Verbatim Text} 
\begin{itemize}
\item If you use the verbatim environment, everything input between the begin and end commands are processed as if by a typewriter. All spaces and new lines are reproduced as given. Any \LaTeX\ command will be ignored and handled as plain text.\\
\item If you use the alltt package, it's almost the same as verbatim except it still process other \LaTeX\ commands inside the begin and end commands.
\end{itemize}

}
%%---------------------------------------------------------------------------Slide 15

%%-----------------------------------------------------------------------------
\frame {\frametitle{Verbatim Text} 

\begin{figure}[h!]
\centering
\includegraphics[width=9cm]{verba.jpg}
\end{figure}
}
%%--------------------------------------------------------------------------- 16
\frame { \frametitle{Comments}
\begin{itemize}
\item Special character \%, that will comment out all the rest of the line after itself
\item An environment called \texttt{comment} will comment out everything within itself. \\
(Require ``verbatim package'')
\end{itemize}

\begin{figure}[h!]
\centering
\includegraphics[width=10cm]{comment.jpg}
\end{figure}

}

%%--------------------------------------------------------------------------- 17

\frame {
\Huge Thank you!
}
%%-----------------------------------------------------------------------------


\end{document}
