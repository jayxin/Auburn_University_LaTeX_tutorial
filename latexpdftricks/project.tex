\documentclass[article]{beamer}

\usetheme{Warsaw}
\setbeamertemplate{footline}[frame number]
\usefonttheme[]{serif}

%% Font packages
\usepackage{lmodern}
\newcommand{\tsans}{\tiny\sffamily}

\usepackage{geometry}
\geometry{verbose,letterpaper}
\usepackage{movie15}

%% Math packages
\usepackage{amsmath}
\usepackage{bm}
%% Table packages
\usepackage{multirow}
\usepackage{booktabs}
\usepackage{graphicx}
\usepackage{overpic}
\usepackage[format=hang]{caption}
\usepackage[format=hang,font=scriptsize]{subcaption}
\usepackage{pdfpages}
\usepackage{lscape}
\usepackage{cooltooltips}
\usepackage{verbatim}
\newcommand{\ic}[2]{\texttt{$\backslash$includegraphics[#1]\{#2\}}}
\usepackage{hyperref}
\synctex=1

\title{\textsc{PDF tricks in \LaTeX}}
\author[Timmy Sanders]{\textsc{Timmy Sanders}}
\institute{\textsc{Department of Mathematics and Statistics\\Auburn University}}
\date{\textsc{\scriptsize{\today}}}

\AtBeginSection[]
{
  \begin{frame}{Outline}
    \tableofcontents[currentsection]
  \end{frame}
}

\begin{document}

\maketitle

\begin{frame}[fragile]
\frametitle{Outline}
\tableofcontents
\end{frame}

\section{Introduction}
\subsection{Overview}

\begin{frame}[fragile]{Overview}

\begin{block}{Overview}

The Portable Document Format (PDF) has many end uses. Most users only consider PDF for static printable documents. However, Adobe has integrated many of the multimedia aspects of Flash.

\end{block}

\end{frame}

\subsection{Helpful Links and Examples}

\begin{frame}[fragile]{Helpful Links}

\begin{block}{Links}

\begin{itemize}

\item{\url{http://www.uoregon.edu/~noeckel/index.html}}
\item{\url{http://www.ctan.org/tex}}
\item{\url{http://emacsworld.blogspot.com}}

\end{itemize}

\end{block}

\end{frame}

\begin{frame}[fragile]{Examples}

\begin{block}{Examples}

\begin{itemize}

\item{\url{http://www.tug.org/texshowcase/#dynamics}}
\item{\url{http://www.adobe.com/products/acrobat/readermain.html}}

\end{itemize}

\end{block}

\end{frame}

\subsection{Packages}

\begin{frame}[fragile]{Packages}

\begin{block}{Basics}

The packages needed to utilize the dynamic features of Portable Document Format (PDF) include:

\begin{itemize}

\item hyperref
\item geometry
\item movie15
\item animate
\item cooltooltips

\end{itemize}

\end{block}

\end{frame}

\section{Balloons}
\subsection{Overview}

\begin{frame}[fragile]{Overview}

\begin{block}{Basics}

The packages needed to include cooltooltip commands:

\begin{itemize}
\item cooltooltips
\item hyperref
\end{itemize}

\end{block}

\end{frame}
\begin{frame}[fragile]{Overview}

\begin{block}{Overview}

The package cooltooltips is a macro that allows the developer of PDFs to include different interactive features. There are two commands that initiate cooltooltips:
\begin{itemize}
\item \verb! \cooltooltip!
\item \verb! \cooltooltiptoggle!
\end{itemize}

\end{block}

\end{frame}

\subsection{Syntax}

\begin{frame}[fragile]{Syntax}

\begin{block}{Commands}

\begin{itemize}
\item \verb! \cooltooltip[<popup color>][<link color>]{<subject>}{<message>}{<url>}{<tooltip>}{<text>}!
\item \verb! \cooltooltiptoggle{<text>}!
\end{itemize}

\end{block}

\end{frame}

\begin{frame}[fragile]{Syntax}
\centering
\begin{tabular}{|l|r|}
\hline
\multicolumn{2}{|c|}{Syntax for cooltooltip}\\
\hline
popup color & color of box around text\\
\hline
link color & color of box around hyperlink\\
\hline
subject & subject in popup window\\
\hline
message & text in popup window\\
\hline
url & hyperlink\\
\hline
tooltip & text displayed while mouse over link\\
\hline
text & text in hyperlink\\
\hline
\multicolumn{2}{|c|}{Syntax for cooltooltiptoggle}\\
\hline
text & text in toggle button\\
\hline
\end{tabular}

\end{frame}


\begin{frame}[fragile]{Example}

\begin{block}{Syntax}

\begin{verbatim}

\cooltooltip
[0 0 1]
\{Example}
\{This is an example of a cool tooltip. Pretty cool, eh?}
\{http://www.ctan.org/}{Visit CTAN on the Web}
\{This text\strut} is an example.

\end{verbatim}

\end{block}

\end{frame}


\section{Video}
\subsection{Overview}

\begin{frame}[fragile]{Overview}

\begin{block}{Basics}

The packages needed to include a mp4 or Flash multimedia are:

\begin{itemize}
\item movie15
\item hyperref
\end{itemize}

\end{block}

\end{frame}

\subsection{Syntax}

\begin{frame}[fragile]{Syntax}

\begin{block}{Syntax}
\LaTeX {} processes the syntax for multimedia within the figure environment.
\begin{verbatim}

\begin{figure}[h]
\includemovie[poster,text={text here}]{cm}{cm}{name of file}
\end{figure}

\end{verbatim}

\end{block}

\end{frame}

\section{Animation}
\subsection{Overview}

\begin{frame}[fragile]{Overview}

\begin{block}{Basics}
In a PDF, the movie15 package creates video via frames of image files. Similarly, a PDF can use vector graphics for the frames to create motion within a document. The packages needed are:
\begin{itemize}
\item animate
\item graphicx
\end{itemize}
\end{block}

\end{frame}

\subsection{Syntax}

\begin{frame}[fragile]{Syntax}

\begin{block}{Environment}

\begin{itemize}
\item \verb! \begin{animateinline}[<options>]{<frame rate>} ... \end{animateinline}!
\end{itemize}

\end{block}

\end{frame}

\begin{frame}[fragile]{Syntax}

\begin{block}{Command}

\begin{itemize}
\item \verb! \animategraphics[<options>]{<frame rate>}{<file basename>}{<first>}{<last>}!
\end{itemize}

\end{block}

\end{frame}

\begin{frame}[fragile]{Example}

\begin{block}{Syntax}
This animation is of the function $y=\exp(x)$.
\begin{verbatim}
\animategraphics[controls, loop,timeline=timeline.txt]{4}{exp_}{0}{8}!
\end{verbatim}

\end{block}

\end{frame}


\section{PDFComments}
\subsection{Overview}

\begin{frame}[fragile]{}

\begin{block}{Overview}
A PDF has several different styles of comments to add various information in a document.
\end{block}

\end{frame}

\subsection{Syntax}

\begin{frame}[fragile]{Syntax}

\begin{block}{Environment}

\begin{itemize}
\item \verb! \begin{pdfsidelinecomment}[<options>]{<comments>} ... \end{pdfsidelinecomment}!
\end{itemize}

\end{block}

\end{frame}

\begin{frame}[fragile]{Syntax}

\begin{block}{Commands}

\begin{itemize}
\item \verb! \pdfcomment[<options>]{<comment>}!
\item \verb! \pdfmargincomment[<options>]{<comment>}!
\item \verb! \pdfmarkup[<options>]{<markup text>}{<comment>}!
\item \verb! \pdfsquarecomment[<options>]{<comment>}!
\item \verb! \pdfcirclecomment[<options>]{<comment>}!
\item \verb! \pdflinecomment[<options>]{<comment>}!
\end{itemize}

\end{block}

\end{frame}

\section{Conclusion}

\begin{frame}[fragile]{}

\begin{block}{Conclusion}

Multimedia is imbedded into a PDF to transform a traditional static document into a dynamical document. The interaction of the document and the user, provides the developer the ability to include hyperlinks, popups, descriptions, etc. to enrich the document style.

\end{block}

\end{frame}

\subsection{Questions}

\begin{frame}[fragile]{}

\centering
{\Huge{\bfseries QUESTIONS?}}

\end{frame}


\end{document} 