\documentclass[10pt]{beamer}
\title{An Introduction to JabRef}
\author{Brian Lilly}
\date{July 19, 2010}

\usetheme{Warsaw}
\setbeamertemplate{footline}[frame number]

\usefonttheme[]{serif}
\usepackage{amsmath, latexsym, color, graphicx}
\usepackage{epsf, epsfig,subfigure}
\usepackage{amsfonts,multicol}
\usepackage{hyperref}

\begin{document}
\maketitle
\begin{frame}
\frametitle{What is JabRef?}
\begin{itemize}
\item Organizes BibTeX files in databases so they can be more easily manipulated
\item Java - Not OS Dependent
\end{itemize}
\end{frame}

\begin{frame}
\frametitle{What is BibTeX?}
\begin{itemize}
\item ``Reference Management Software"
\item Contains information needed to cite a source
	\begin{itemize}
	\item Author(s)
	\item Publication
	\item Volumes, Page Numbers, etc.
	\end{itemize}
\end{itemize}
\end{frame}

\begin{frame}
\frametitle{Two Ways to Use JabRef}
\begin{itemize}
\item \url{http://jabref.sourceforge.net/}
\item Standalone version
\item Web Version
	\begin{itemize}
	\item Allows use of JabRef without installing
	\end{itemize}
\end{itemize}
\end{frame}

\begin{frame}
\frametitle{Getting BibTeX Files}
\begin{itemize}
\item \href{http://apps.isiknowledge.com/WOS_GeneralSearch_input.do?product=WOS&search_mode=GeneralSearch&SID=4El@7A2mGNLCA3odgOE&preferencesSaved=&highlighted_tab=WOS}{ISI Web of Science}
	\begin{itemize}
	\item Search
	\item Check boxes of articles
	\item ``Add to Marked List"
	\item Check all of the boxes
	\item Change ``Other Reference Software" to ``Bibtex" and ``Save to File"
	\end{itemize}
\end{itemize}
\end{frame}

\begin{frame}
\frametitle{Entry Editor}
\begin{itemize}
\item Can change what columns are shown in the main window
	\begin{itemize}
	\item Entry Table Columns (Options$\Rightarrow$Preferences$\Rightarrow$Entry Table Columns)
	\end{itemize}
\item Uppercase letters: \{U\}ppercase or Options$\Rightarrow$Preferences$\Rightarrow$General$\Rightarrow$File$\Rightarrow$``Store the following fields with braces around capital letters"
\end{itemize}
\end{frame}

\begin{frame}
\frametitle{Linking}
\begin{itemize}
\item Linking to a url (Entry Editor$\Rightarrow$General$\Rightarrow$url)
\item Linking to a pdf (Entry Editor$\Rightarrow$General$\Rightarrow$File$\Rightarrow$+)
\item Can link to pdfs relative to the Bibtex database file's location
	\begin{itemize}
	\item Set a File Directory (File$\Rightarrow$Database Properties$\Rightarrow$File directory)
	\item Put in the directory for the folder the database is in
	\item Where the full path for the pdf was before: .{\textbackslash}nameofthefile.pdf or .{\textbackslash}subfolder{\textbackslash}nameofthefile.pdf
	\item Then, when the database and articles are moved, change the File directory to where they have been moved
	\end{itemize}
\end{itemize}
\end{frame}

\begin{frame}
\frametitle{Searching}
\begin{itemize}
\item Search button: Magnifying glass in Toolbar or ctrl-f
	\begin{itemize}
	\item Can use logical operators:
	\item field = keyword
	\item field = ``multiple word phrase"
	\item field$|$anotherfield = keyword and ``multiple word phrase"
	\item (field = keyword or field$|$anotherfield = ``multiple word phrase") and not field = keyword
	\end{itemize}
\end{itemize}
\end{frame}

\begin{frame}
\frametitle{Strings}
\begin{itemize}
\item To open string editor: BibTeX$\Rightarrow$Edit Strings
\item Strings are composed of a ``name" and a ``content"
	\begin{itemize}
	\item The name would be some text enclosed by \#s, example: \#asdf\#
	\item The content would then be some other text, example: Left Hand Home Row
	\end{itemize}
\item Anytime \#asdf\# appears in a field, it will reference the text Left Hand Home Row
\item To have strings applied for any non-standard Bibtex fields
	\begin{itemize}	
	\item Options$\Rightarrow$Preferences$\Rightarrow$File$\Rightarrow$``Resolve strings for all fields except:"
	\end{itemize}
\end{itemize}
\end{frame}

\begin{frame}
\frametitle{Journal Abbreviations}
\begin{itemize}
\item Set up a journal 
abbreviation list (Options$\Rightarrow$Manage journal abbreviations)
	\begin{itemize}
	\item Create a file
	\item Add abbreviations exactly (Case-sensitive and with curly brackets)
	\end{itemize}
\item If everything is moved to another computer, import abbreviations file into JabRef as an ``Existing File"
\end{itemize}
\end{frame}

\begin{frame}
\frametitle{Groups}
\begin{itemize}
\item Groups button: Gray squares button in toolbar
\item Add new groups either manually or by using a search method
\item Can select multiple groups and use Settings$\Rightarrow$Union and Settings$\Rightarrow$Intersection
\end{itemize}
\end{frame}

\begin{frame}
\frametitle{BibTeX Keys}
\begin{itemize}
\item These are used to cite individual sources within a LaTeX document
	\begin{itemize}
	\item {\textbackslash}cite\{BibTeX Key\}
	\end{itemize}
\item Generate new key by selecting the article in the main window and hitting ctrl-g or clicking the wand button
\item Options$\Rightarrow$Preferences$\Rightarrow$BibTeX key generator
\item Made up of field markers: [field]
\item The default pair of field markers is [auth][year]
	\begin{itemize}
	\item Makes a BibTeX key out of the last name of the first author, [auth], followed by the year of publication, [year]
	\end{itemize}
\item See: \url{http://jabref.sourceforge.net/help/LabelPatterns.php} for a list of some ``special" field markers
	\begin{itemize}
	\item ``Special" because they only use a portion of a field ([auth], above, is one of these)
	\end{itemize}
\end{itemize}
\end{frame}

\begin{frame}
\frametitle{Other Stuff}
\begin{itemize}
\item Can use plugins to extend JabRef's abilities
\item Can add custom tabs to the Entry Editor
	\begin{itemize}
	\item Options$\Rightarrow$Set up general fields
	\item One tab on each line: NewTabName:field1;field2;field3;etc.
	\end{itemize}
\item Probably a lot of other things that I don't even know about
\item See \url{http://jabref.sourceforge.net/help/Contents.php} for more help
	\begin{itemize}
	\item This is where I got most of the information for this presentation
	\end{itemize}
\end{itemize}
\end{frame}

\begin{frame}
\frametitle{?}
\centering\Huge ?
\end{frame}

\end{document}