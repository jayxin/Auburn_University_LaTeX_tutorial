\documentclass{beamer}
\usetheme[hideothersubsections]{AUTheme}
\usefonttheme[]{serif}

\usepackage{pgf,pgfpages,epic,eepic}
%\pgfpagesuselayout{4 on 1}[letterpaper,landscape,border
%shrink=0.5in]

\usepackage{graphicx}
\usepackage{amsmath, latexsym, color, amssymb, here}
\usepackage{epsf, epsfig, pifont,tikz,subfigure}
\usepackage{graphics, calrsfs}
\usepackage{times}
\usepackage{fancybox,calc}
\usepackage{palatino,mathpazo}
%
%\documentclass{article}
%\usepackage{graphicx,anysize}
\usepackage{pgfplots,verbatim}
% ----------------------------------------------------------------
% ----------------------------------------------------------------
%\marginsize{1in}{1in}{1in}{1in}%
%\renewcommand{\baselinestretch}{1.3}
\begin{document}

\begin{frame}\frametitle{Intro to PGFPlots and Ti\emph{k}Z}

\begin{itemize}
	\item 
draws high-quality function plots in normal or logarithmic scaling with a user-friendly
interface directly in TEX. 
\item user supplies axis labels, legend entries and the plot coordinates for one
or more plots 
\item pgfplots applies axis scaling, computes any logarithms and axis ticks and draws
the plots, supporting line plots, scatter plots, piecewise constant plots, bar plots, area plots, mesh, and
surface plots and some more. 
\item based on Till Tantau's package pgf/TikZ.
\end{itemize}
\end{frame}

\noindent
%$\sin x \approx x$
\frame {\frametitle{stem plot}

\begin{center}

\input test5.tikz
\end{center}
}

\begin{frame}[fragile]

\tiny
\verbatiminput{test5.tikz}

\end{frame}

\frame{
\input test4.tikz
}


\frame{
\begin{tikzpicture}
\begin{axis}[
xlabel=Cost,
ylabel=Error]
\addplot[color=red,mark=x] coordinates {
(2,-2.8559703)
(3,-3.5301677)
(4,-4.3050655)
(5,-5.1413136)
(6,-6.0322865)
(7,-6.9675052)
(8,-7.9377747)
};
\end{axis}
\end{tikzpicture}
}

\frame{
% Preamble: \pgfplotsset{width=7cm,compat=newest}
\begin{tikzpicture}
\begin{axis}[
axis equal=true,
axis x line=middle,
axis y line=middle]
%xlabel=$x$,
%ylabel={$f_0.5(x)$}
]
% use TeX as calculator:
\addplot[mark=none,smooth,domain=-1.5:1.5] {x^2*(abs(x)<0.5) + (0.5^2 + 2*0.5*(abs(x)-0.5))*(1-(abs(x)<0.5))};
\end{axis}
\end{tikzpicture}
}

\frame{
\begin{verbatim}
\begin{tikzpicture}
\begin{axis}[
axis equal=true,
axis x line=middle,
axis y line=middle]
]
% use TeX as calculator:
\addplot[mark=none,smooth,domain=-1.5:1.5] {x^2*(abs(x)<0.5) + (0.5^2 + 2*0.5*(abs(x)-0.5))*(1-(abs(x)<0.5))};
\end{axis}
\end{tikzpicture}
\end{verbatim}
}

\frame{
\begin{tikzpicture}
\begin{axis}[
axis x line=middle,
axis y line=middle]
% use TeX as calculator:
%\addplot[mark=none,domain=-4:4,samples=200] {x^2 - x +4};
\addplot[mark=none] coordinates {
(-7,-6) 
(-4,-6) 
(-4,-3) 
(-2,-3) 
(-2,-1.5) 
(-1,-1.5) 
(-1,-0.75)
(-0.5,0) 
(0.5,0)  
};
\end{axis}
\end{tikzpicture}
}

\end{document}
% ----------------------------------------------------------------
