%% LyX 1.6.6.1 created this file.  For more info, see http://www.lyx.org/.
%% Do not edit unless you really know what you are doing.
\documentclass[english]{beamer}
\renewcommand{\familydefault}{\rmdefault}
\usepackage[T1]{fontenc}
\usepackage[latin9]{inputenc}
\setcounter{secnumdepth}{3}
\setcounter{tocdepth}{3}
\usepackage{graphicx}

\makeatletter

%%%%%%%%%%%%%%%%%%%%%%%%%%%%%% LyX specific LaTeX commands.
\providecommand{\LyX}{L\kern-.1667em\lower.25em\hbox{Y}\kern-.125emX\@}
\DeclareRobustCommand*{\lyxarrow}{%
\@ifstar
{\leavevmode\,$\triangleleft$\,\allowbreak}
{\leavevmode\,$\triangleright$\,\allowbreak}}

%%%%%%%%%%%%%%%%%%%%%%%%%%%%%% Textclass specific LaTeX commands.
 % this default might be overridden by plain title style
 \newcommand\makebeamertitle{\frame{\maketitle}}%
 \AtBeginDocument{
   \let\origtableofcontents=\tableofcontents
   \def\tableofcontents{\@ifnextchar[{\origtableofcontents}{\gobbletableofcontents}}
   \def\gobbletableofcontents#1{\origtableofcontents}
 }
 \makeatletter
 \long\def\lyxframe#1{\@lyxframe#1\@lyxframestop}%
 \def\@lyxframe{\@ifnextchar<{\@@lyxframe}{\@@lyxframe<*>}}%
 \def\@@lyxframe<#1>{\@ifnextchar[{\@@@lyxframe<#1>}{\@@@lyxframe<#1>[]}}
 \def\@@@lyxframe<#1>[{\@ifnextchar<{\@@@@@lyxframe<#1>[}{\@@@@lyxframe<#1>[<*>][}}
 \def\@@@@@lyxframe<#1>[#2]{\@ifnextchar[{\@@@@lyxframe<#1>[#2]}{\@@@@lyxframe<#1>[#2][]}}
 \long\def\@@@@lyxframe<#1>[#2][#3]#4\@lyxframestop#5\lyxframeend{%
   \frame<#1>[#2][#3]{\frametitle{#4}#5}}
 \makeatother
 \def\lyxframeend{} % In case there is a superfluous frame end

%%%%%%%%%%%%%%%%%%%%%%%%%%%%%% User specified LaTeX commands.
\usetheme{Madrid}
\AtBeginSection[]
{
  \begin{frame}{Outline}
    \tableofcontents[currentsection]
  \end{frame}
}

\makeatother

\usepackage{babel}

\begin{document}

\title{\LyX{}: The WYSIWYM Document Processor}


\author{Robert Thetford, Jr.}


\date{July 9, 2010}

\makebeamertitle

\lyxframeend{}\lyxframe{Outline}

\tableofcontents{}


\lyxframeend{}


\lyxframeend{}\section{What is \protect\LyX{}?}


\lyxframeend{}\lyxframe{What Is \LyX{}?}

\setbeamercovered{dynamic}

\LyX{} is document processor that works as a front end to \LaTeX{}.
It allows for instant previewing of the document from the input side
of the process, creating a near WYSIWYG version of \LaTeX{}, coined
by \LyX{} users as WYSIWYM.

We'll overview three of the common document classes:


\pause{}
\begin{enumerate}
\item Article
\item Letter
\item Beamer
\end{enumerate}

\lyxframeend{}


\lyxframeend{}\section{Article}


\lyxframeend{}\lyxframe{Article}

\textsf{Article} is the default document class when \LyX{} is first
opened. The structure of sections, titles, and lists within the document
works the same as in \LaTeX{}, but are entered differently. This will
be explained in the next slide. These {}``paragraph styles'' are
refered to, in \LyX{}, as environments. \textsf{Article} allows for
other typical scientific document environments such as tables, figures,
math expressions, etc. 


\lyxframeend{}


\lyxframeend{}\subsection{Sections, Titles, and Lists}


\lyxframeend{}\lyxframe{Article}


\framesubtitle{Sections, Titles, and Lists}

\setbeamercovered{dynamic}

Sections are very straightforward.


\pause{}
\begin{itemize}
\item Click the cursor to the desired line in the document
\end{itemize}

\pause{}
\begin{itemize}
\item Click the \textsf{Environment} dropbox (the leftmost element on the
buttonbar)
\end{itemize}

\pause{}
\begin{itemize}
\item Select \textsf{Section}, \textsf{Subsection}, \textsf{Subsubsection},
\textsf{Paragraph}, or \textsf{Subparagraph}
\end{itemize}

\pause{}

This automatically makes the selected line into the desired environment.
Using this same method, a selected line can be made into an unnumbered
section (\textsf{Section{*}}), \textsf{Title}, \textsf{Author}, \textsf{Enumerate}
list, \textsf{Itemize} list, etc. All these environments are selectable
in the dropbox. 


\lyxframeend{}


\lyxframeend{}\lyxframe{Article}


\framesubtitle{Sections, Titles, and Lists}

\includegraphics[width=0.9\textwidth]{\string"LyX Sections\string".eps}


\lyxframeend{}


\lyxframeend{}\subsection{Tables}


\lyxframeend{}\lyxframe{Article}


\framesubtitle{Tables}

\setbeamercovered{dynamic}

Tables are very straightforward. 


\pause{}
\begin{itemize}
\item Click the cursor to the desired line in the document
\end{itemize}

\pause{}
\begin{itemize}
\item Create a new table with \textsf{Insert}\lyxarrow{}\textsf{Table}
\end{itemize}

\pause{}
\begin{itemize}
\item Specify how many columns and rows 
\end{itemize}

\pause{}
\begin{itemize}
\item Click each and edit each individual cell
\end{itemize}

\pause{}
\begin{itemize}
\item Use the Table Toolbar (shows at bottom of screen when inserted table
is left-clicked) to toggle borderlines and designate multicolumns
(multirows not supported by \LyX{})
\end{itemize}

\pause{}


\note[item]{Tables can also be created with the \textsf{Insert table} button}


\lyxframeend{}


\lyxframeend{}\lyxframe{Article}


\framesubtitle{Tables}

\includegraphics[width=0.9\textwidth]{\string"LyX Table\string".eps}


\lyxframeend{}


\lyxframeend{}\subsection{Figures}


\lyxframeend{}\lyxframe{Article}


\framesubtitle{Figures}

\setbeamercovered{dynamic}

Figures are very straightforward. In \LyX{}, they are refered to as
graphics.


\pause{}
\begin{itemize}
\item Click the cursor to the desired line in the document
\end{itemize}

\pause{}
\begin{itemize}
\item Create a new graphic with \textsf{Insert}\lyxarrow{}\textsf{Graphics}
\end{itemize}

\pause{}
\begin{itemize}
\item Specify an image to insert (all known image formats will work)
\end{itemize}

\pause{}
\begin{itemize}
\item Optional: Specify a scale percentage, scale by width, or scale by
height value
\end{itemize}

\pause{}


\note[item]{Graphics can also be inserted using the \textsf{Insert graphics}
button.}


\lyxframeend{}


\lyxframeend{}\lyxframe{Article}


\framesubtitle{Figures}

\includegraphics[width=0.9\textwidth]{\string"LyX Graphic\string".eps}


\lyxframeend{}


\lyxframeend{}\subsection{Math}


\lyxframeend{}\lyxframe{Article}


\framesubtitle{Math}

\setbeamercovered{dynamic}

Math is very straightforward.


\pause{}
\begin{enumerate}
\item Inline Equations


\pause{}
\begin{itemize}
\item Click the cursor to the desired line in the document
\end{itemize}

\pause{}
\begin{itemize}
\item Create a new math formula with \textsf{Insert}\lyxarrow{}\textsf{Math}\lyxarrow{}\textsf{Inline
Formula}
\end{itemize}

\pause{}
\begin{itemize}
\item Enter \LaTeX{} math commands or \LyX{} shorthand commands
\end{itemize}

\pause{}
\begin{itemize}
\item Hit \textsf{Esc} to end the equation
\end{itemize}

\pause{}

Displayed Equations are quite similar:

\item Displayed Equations
\end{enumerate}

\pause{}
\begin{itemize}
\item Create a new math formula with \textsf{Insert}\lyxarrow{}\textsf{Math}\lyxarrow{}\textsf{Display
Formula}
\end{itemize}

\pause{}
\begin{itemize}
\item Give the equation a number label with \textsf{Edit}\lyxarrow{}\textsf{Math}\lyxarrow{}\textsf{Number
Whole Formula }(for a single line equation)
\end{itemize}

\lyxframeend{}


\lyxframeend{}\lyxframe{Article}


\framesubtitle{Math}

\includegraphics[width=0.9\textwidth]{\string"LyX Math\string".eps}


\lyxframeend{}


\lyxframeend{}\section{Letter}


\lyxframeend{}\lyxframe{Letter}

\LyX{} allows for a simple but professional looking \textsf{Letter}
document. \textsf{Letter} can be written from scratch just like \textsf{Article},
except that there are different environments available. However, an
advantage to using \LyX{} is that there are some templates available
to make creating documents even easier. One template included with
\LyX{} is a \textsf{Letter} template.


\lyxframeend{}


\lyxframeend{}\subsection{Template}


\lyxframeend{}\lyxframe{Letter}


\framesubtitle{Template}

\includegraphics[width=0.9\textwidth]{\string"LyX letter template\string".eps}

%
\begin{figure}


\caption{Letter Template\label{fig:Letter-Template}}



\end{figure}



\lyxframeend{}


\lyxframeend{}\section{Beamer}


\lyxframeend{}\lyxframe{Beamer}

\textsf{Beamer} is the slide presentation document class. From a \LaTeX{}
user standpoint, the input process in \LyX{} for a \textsf{Beamer}
document is a little odd at first, but simple after repeated use.
Like \textsf{Article} and \textsf{Letter}, all the document class
specific environments can be found in the \textsf{Environment} dropbox.
The following screenshot is for the second frame of this presentation.


\lyxframeend{}


\lyxframeend{}\subsection{Example}


\lyxframeend{}\lyxframe{Beamer}


\framesubtitle{Example}

\includegraphics[width=0.9\textwidth]{\string"LyX Beamer\string".eps}


\lyxframeend{}


\lyxframeend{}\section{Conclusion}


\lyxframeend{}\lyxframe{Conclusion}

In conclusion, \LyX{} is simple and hard?
\begin{itemize}
\item Creating documents that work within the default parameters is \emph{very}
simple.
\item Creating documents that require much fine tuning or more complex elements
is difficult.
\end{itemize}
Questions?


\lyxframeend{}
\end{document}
